\chapter{Ứng dụng dự đoán từ tiếp theo}

%%%%%% NP
% \section{Dự đoán từ tiếp theo}
% \subsection{Đặt vấn đề}
\section{Đặt vấn đề}
\begin{itemize}
    \item Trong rất nhiều các công cụ tìm kiếm, cũng như các phần mềm cho ứng dụng tìm kiếm hiện nay, chúng ta có thể dễ dàng nhận ra được một chức năng rất phổ biến, đó chính là đưa ra những đề xuất về từ ngữ tiếp theo dựa vào những từ vừa được gõ. Từ đó chúng ta có bài toán làm thế nào để tìm ra đâu là những từ có xác suất xuất hiện tiếp theo cao nhất.
    \item Có rất nhiều những phương pháp có thể được áp dụng trong việc giải quyết bài toán trên một cách hiệu quả chẳng hạn như Deep Learning. Tuy nhiên, chúng ta cũng có thể giải quyết bài toán này (một cách đơn giản, song không tối ưu như DL) bằng chuỗi Markov.
\end{itemize}

% \subsection{Chuẩn bị và xử lí dữ liệu}
\section{Chuẩn bị và xử lí dữ liệu}
\begin{itemize}
    \item Đầu tiên, từ một tập dữ liệu (dataset) gồm những từ, câu đã được tìm kiếm (lấy từ database của công cụ tìm kiếm) hoặc từ những từ đã được gõ (lấy từ database của phần mềm bàn phím điện thoại như LabanKey), chúng ta sẽ loại bỏ các kí tự đặt biệt (dấu câu,..), sau đó phân tách từng câu ra thành cụm lần lượt có 1, 2 và 3 từ, sau đó xây dựng ma trận xác suất (sẽ được trình bày ở phần sau). 
    \item Để dễ hình dung cách hoạt động của thuật toán, trong quy mô bài báo cáo này, chúng tôi xin được minh hoạt bằng 1 dataset lấy từ một vài câu comment trên Facebook như sau:

    \begin{center}
    "Tin này là tin chuẩn chưa anh?" \\
    "Này là tin chuẩn em ạ!" \\
    "Đừng spam tin này nữa em nhé!" \\
    "Tin này chưa chuẩn em nhé!" \\
    "Anh sẽ chặn các bạn spam tin này." 
    \end{center}

    \item Phân tách thành các cụm 1 từ, ta được các cụm sau: 
    'tin', 'này', 'là', 'chuẩn', 'chưa', 'anh', 'em', 'ạ', 'đừng', 'spam', 'nữa', 'nhé', 'sẽ', 'chặn', 'các', 'bạn'.
    \item Phân tách thành các cụm 2 từ, ta được các cụm sau: 'tin này', 'này là', 'là tin', 'tin chuẩn', 'chuẩn chưa', 'chưa anh', 'chuẩn em', 'em ạ', 'đừng spam', 'spam tin', 'này nữa', 'nữa em', 'em nhé', 'này chưa', 'chưa chuẩn', 'anh sẽ', 'sẽ chặn', 'chặn các', 'các bạn', 'bạn spam'.
    \item Phân tách thành các cụm 3 từ, ta được các cụm sau: 'tin này là', 'này là tin', 'là tin chuẩn', 'tin chuẩn chưa', 'chuẩn chưa anh', 'tin chuẩn em', 'chuẩn em ạ', 'đừng spam tin', 'spam tin này', 'tin này nữa', 'này nữa em', 'nữa em nhé', 'tin này chưa', 'này chưa chuẩn', 'chưa chuẩn em', 'chuẩn em nhé', 'anh sẽ chặn', 'sẽ chặn các', 'chặn các bạn', 'các bạn spam', 'bạn spam tin'.

\end{itemize}

% \subsection{Xây dựng ma trận chuyển tiếp}
\section{Xây dựng ma trận chuyển tiếp}
\begin{itemize}
    \item Từ những cụm từ chúng ta đã phân tách ở phần trên, chúng ta có thể xây dựng các ma trận chuyển trạng thái biểu diễn xác suất những từ sẽ xuất hiện tiếp theo dựa vào 1 hoặc 2 hoặc 3 từ trước đó. 
    \item Ma trận trạng thái sẽ được định nghĩa như sau: phần tử hàng $i$ cột $j$ sẽ biểu diễn xác suất cho sự xuất hiện của trạng thái tiếp theo $j$ tương ứng với trạng thái hiện tại $i.$ Xác suất đó sẽ được tính bằng số lần xuất hiện của trạng $j$ sau trạng thái $i$ chia cho tổng số lần xuất hiện của các trạng thái. Ví dụ cho ma trận chuyển tiếp với các trạng thái hiện tại là 1 từ:
    $$
    \kbordermatrix{ & \text{tin} & \text{này} & \text{là} & \text{chuẩn} & \text{chưa} & \text{anh} & \text{em} & \text{ạ} & \text{đừng} & \text{spam} & \text{nữa} & \text{nhé} & \text{sẽ} & \text{chặn} & \text{các} & \text{bạn} \\
      \text{tin} & 0 & 2/3 & 0 & 1/3 & 0 & 0 & 0 & 0 & 0 & 0 & 0 & 0 & 0 & 0 & 0 & 0\\
      \text{này} & 0 & 0 & 1/2 & 0 & 1/4 & 0 & 0 & 0 & 0 & 0 & 1/4 & 0 & 0 & 0 & 0 & 0\\
      \text{là} & 1 & 0 & 0 & 0 & 0 & 0 & 0 & 0 & 0 & 0 & 0 & 0 & 0 & 0 & 0 & 0\\
      \text{chuẩn} & 0 & 0 & 0 & 0 & 1/3 & 0 & 2/3 & 0 & 0 & 0 & 0 & 0 & 0 & 0 & 0 & 0\\
      \text{chưa} & 0 & 0 & 0 & 1/2 & 0 & 1/2 & 0 & 0 & 0 & 0 & 0 & 0 & 0 & 0 & 0 & 0\\
      \text{anh} & 0 & 0 & 0 & 0 & 0 & 0 & 0 & 0 & 0 & 0 & 0 & 0 & 1 & 0 & 0 & 0\\
      \text{em} & 0 & 0 & 0 & 0 & 0 & 0 & 0 & 1/3 & 0 & 0 & 0 & 2/3 & 0 & 0 & 0 & 0\\
      \text{ạ} & 0 & 0 & 0 & 0 & 0 & 0 & 0 & 0 & 0 & 0 & 0 & 0 & 0 & 0 & 0 & 0\\
      \text{đừng} & 0 & 0 & 0 & 0 & 0 & 0 & 0 & 0 & 0 & 1 & 0 & 0 & 0 & 0 & 0 & 0\\
      \text{spam} & 1 & 0 & 0 & 0 & 0 & 0 & 0 & 0 & 0 & 0 & 0 & 0 & 0 & 0 & 0 & 0\\
      \text{nữa} & 0 & 0 & 0 & 0 & 0 & 0 & 1 & 0 & 0 & 0 & 0 & 0 & 0 & 0 & 0 & 0\\
      \text{nhé} & 0 & 0 & 0 & 0 & 0 & 0 & 0 & 0 & 0 & 0 & 0 & 0 & 0 & 0 & 0 & 0\\
      \text{sẽ} & 0 & 0 & 0 & 0 & 0 & 0 & 0 & 0 & 0 & 0 & 0 & 0 & 0 & 1 & 0 & 0\\
      \text{chặn} & 0 & 0 & 0 & 0 & 0 & 0 & 0 & 0 & 0 & 0 & 0 & 0 & 0 & 0 & 0 & 0\\
      \text{các} & 0 & 0 & 0 & 0 & 0 & 0 & 0 & 0 & 0 & 0 & 0 & 0 & 0 & 0 & 0 & 1\\
      \text{bạn} & 0 & 0 & 0 & 0 & 0 & 0 & 0 & 0 & 0 & 1 & 0 & 0 & 0 & 0 & 0 & 0} \qquad
    $$
   
    \item Ví dụ cho ma trận chuyển tiếp với trạng thái hiện tại cụm 2 từ:
    \end{itemize}
    $$
    \kbordermatrix{ & \text{tin} & \text{này} & \text{là} & \text{chuẩn} & \text{chưa} & \text{anh} & \text{em} & \text{ạ} & \text{đừng} & \text{spam} & \text{nữa} & \text{nhé} & \text{sẽ} & \text{chặn} & \text{các} & \text{bạn} \\
      \text{tin này} & 0 & 0 & 1/3 & 0 & 1/3 & 0 & 0 & 0 & 0 & 0 & 1/3 & 0 & 0 & 0 & 0 & 0\\
      \text{này là} & 1 & 0 & 0 & 0 & 0 & 0 & 0 & 0 & 0 & 0 & 0 & 0 & 0 & 0 & 0 & 0\\
      \text{là tin} & 0 & 0 & 0 & 1 & 0 & 0 & 0 & 0 & 0 & 0 & 0 & 0 & 0 & 0 & 0 & 0\\
      \text{tin chuẩn} & 0 & 0 & 0 & 0 & 1/2 & 0 & 1/2 & 0 & 0 & 0 & 0 & 0 & 0 & 0 & 0 & 0\\
      \text{chuẩn chưa} & 0 & 0 & 0 & 0 & 0 & 1 & 0 & 0 & 0 & 0 & 0 & 0 & 0 & 0 & 0 & 0\\
      \text{chưa anh} & 0 & 0 & 0 & 0 & 0 & 0 & 0 & 0 & 0 & 0 & 0 & 0 & 0 & 0 & 0 & 0\\
      \text{chuẩn em} & 0 & 0 & 0 & 0 & 0 & 0 & 0 & 1/2 & 0 & 0 & 0 & 1/2 & 0 & 0 & 0 & 0\\
      \dots & \dots & \dots & \dots & \dots & \dots & \dots & \dots & \dots & \dots & \dots & \dots & \dots & \dots & \dots & \dots & \dots \\
      \text{sẽ chặn} & 0 & 0 & 0 & 0 & 0 & 0 & 0 & 0 & 0 & 0 & 0 & 0 & 0 & 0 & 1 & 0\\
      \text{chặn các} & 0 & 0 & 0 & 0 & 0 & 0 & 0 & 0 & 0 & 0 & 0 & 0 & 0 & 0 & 0 & 1\\
      \text{các bạn} & 0 & 0 & 0 & 0 & 0 & 0 & 0 & 0 & 0 & 1 & 0 & 0 & 0 & 0 & 0 & 0\\
      \text{bạn spam} & 1 & 0 & 0 & 0 & 0 & 0 & 0 & 0 & 0 & 0 & 0 & 0 & 0 & 0 & 0 & 0} \qquad
    $$
    \begin{itemize}
    \item Ví dụ cho ma trận chuyển tiếp với trạng thái hiện tại cụm 3 từ:
    \end{itemize}
    $$
    \kbordermatrix{ & \text{tin} & \text{này} & \text{là} & \text{chuẩn} & \text{chưa} & \text{anh} & \text{em} & \text{ạ} & \text{đừng} & \text{spam} & \text{nữa} & \text{nhé} & \text{sẽ} & \text{chặn} & \text{các} & \text{bạn} \\
      \text{tin này là} & 1 & 0 & 0 & 0 & 0 & 0 & 0 & 0 & 0 & 0 & 0 & 0 & 0 & 0 & 0 & 0\\
      \text{này là tin} & 0 & 0 & 0 & 1 & 0 & 0 & 0 & 0 & 0 & 0 & 0 & 0 & 0 & 0 & 0 & 0\\
      \text{là tin chuẩn} & 0 & 0 & 0 & 0 & 1/2 & 0 & 1/2 & 0 & 0 & 0 & 0 & 0 & 0 & 0 & 0 & 0\\
      \text{tin chuẩn chưa} & 0 & 0 & 0 & 0 & 0 & 1 & 0 & 0 & 0 & 0 & 0 & 0 & 0 & 0 & 0 & 0\\
      \dots & \dots & \dots & \dots & \dots & \dots & \dots & \dots & \dots & \dots & \dots & \dots & \dots & \dots & \dots & \dots & \dots \\
      \text{sẽ chặn các} & 0 & 0 & 0 & 0 & 0 & 0 & 0 & 0 & 0 & 0 & 0 & 0 & 0 & 0 & 0 & 1\\
      \text{chặn các bạn} & 0 & 0 & 0 & 0 & 0 & 0 & 0 & 0 & 0 & 1 & 0 & 0 & 0 & 0 & 0 & 0\\
      \text{các bạn spam} & 1 & 0 & 0 & 0 & 0 & 0 & 0 & 0 & 0 & 0 & 0 & 0 & 0 & 0 & 0 & 0\\
      \text{bạn spam tin} & 0 & 1 & 0 & 0 & 0 & 0 & 0 & 0 & 0 & 0 & 0 & 0 & 0 & 0 & 0 & 0 \\
      \text{spam tin này} & 0 & 0 & 0 & 0 & 0 & 0 & 0 & 0 & 0 & 0 & 0 & 0 & 0 & 0 & 0 & 0\\} \qquad
$$


% \subsection{Tiến hành dự đoán từ tiếp theo}
\section{Tiến hành dự đoán từ tiếp theo}
\begin{itemize}
    \item Đầu tiên, với từ ngữ đầu tiên được nhập (từ bàn phím), ta có thể thành lập 1 vector phân phối các trạng thái hiện tại. Vì mới chỉ có 1 từ nên chúng ta sử dụng ma trận phân phối với các trạng thái hiện tại là 1 từ, ví dụ cụ thể như sau:
    \begin{itemize}
        \item Với từ đầu tiên được nhập là "tin", thứ tự tương ứng với thứ tự các trạng thái trong ma trận chuyển tiếp ở trên, ta được vector phân phối:
    \end{itemize}
    $$
    \pi_0 = 
    \begin{bmatrix}
        1 & 0 & 0 & 0 & 0 & 0 & 0 & 0 & 0 & 0 & 0 & 0 & 0 & 0 & 0 & 0 
    \end{bmatrix} 
    $$
    
    
    \item Từ đó, ta có thể xác định phân phối tiếp theo:
    $$ \pi_1 = \pi_0 P_1 = 
    \begin{bmatrix}
        0 & 2/3 & 0 & 1/3 & 0 & 0 & 0 & 0 & 0 & 0 & 0 & 0 & 0 & 0 & 0 & 0 
    \end{bmatrix}
    $$

    \item Vậy, ta có thể đề xuất những từ có xác suất cao nhất:
    \begin{itemize}
        \item Từ "này" \space với xác suất $\frac{2}{3}.$
        \item Từ "chuẩn" \space với xác suất $\frac{1}{3}.$
        
    \end{itemize}
    
    \item Nếu người dùng đã viết được ít nhất 2 từ, để tăng độ chính xác khi đề xuất từ cũng như đảm bảo một phần yếu tố ngữ nghĩa của câu, ta sử dụng ma trận chuyển tiếp của trạng thái 2 từ và ma trận chuyển trạng thái 3 từ. Từ đó, ta có thể xây dựng thuật toán như sau:

    \begin{itemize}
        \item Với những câu chỉ có một từ, sử dụng ma trận chuyển tiếp với trạng thái 1 từ.
        \item  Với những câu có nhiều hơn 2 từ, lần lượt xét 3 từ cuối của câu xem đã tồn tại thái là 3 từ đấy chưa, nếu có, sử dụng ma trận chuyển tiếp với trạng thái 3 từ, nếu chưa thì chuyển sang xét 2 từ cuối và ma trận chuyển tiếp với trạng thái 2 từ, và cứ như thế cho đến khi còn 1 từ.
        \item Sau khi hoàn thành, tiếp tục cập nhật các trạng thái và xác suất mới vào dataset.
    \end{itemize}
    \item Có thể minh hoạ với ví dụ sau:
    \begin{itemize}
        \item Với cụm từ đã gõ "Anh cho em hỏi tin này là", ta sẽ xét 3 từ cuối là "tin này là", nhận thấy đây là 1 trạng thái trong ma trận chuyển tiếp, áp dụng mô hình ở trên, ta được phân phối cho từ tiếp theo:
        $$ \pi_1 = 
        \begin{bmatrix}
            1 & 0 & 0 & 0 & 0 & 0 & 0 & 0 & 0 & 0 & 0 & 0 & 0 & 0 & 0 & 0 
        \end{bmatrix}
        $$
        \item Từ đó, ta đề xuất từ "tin" \space cho người dùng với xác suất bằng 1.        
    \end{itemize}

    \item Một ví dụ khác:
    \begin{itemize}
        \item Với cụm từ đã gõ "Đây có phải tin chuẩn", ta sẽ xét 3 từ cuối là "phải tin chuẩn", nhận thấy đây không là 1 trạng thái trong ma trận chuyển tiếp, ta xét 2 từ cuối là "tin chuẩn", nhận thấy đây là một trạng thái của ma trận chuyển tiếp 2 từ, áp dụng mô hình ở trên, ta được phân phối cho từ tiếp theo:
        $$ \pi_1 = 
        \begin{bmatrix}
            0 & 0 & 0 & 0 & 1/2 & 0 & 1/2 & 0 & 0 & 0 & 0 & 0 & 0 & 0 & 0 & 0 
        \end{bmatrix}
        $$
        \item Từ đó, ta đề xuất từ "tin" \space cho người dùng với xác suất bằng $0.5$, hoặc từ "em" \space với xác suất bằng $0.5$.        
    \end{itemize}
    
\end{itemize}

% \subsection{Code cơ bản cho mô hình}
\section{Code cơ bản cho mô hình}
% \begin{itemize}
%     \item Giống như đồ thị, để tiết kiệm bộ nhớ thì thay vì sử dụng ma trận kề, các lập trình viên thường sử dụng lớp vector (trong C++). Tương tự với mô hình trên, chúng ta sẽ dùng các dictionary để lưu trữ cấc trạng thấi và tính toán các phân phối. Chi tiết code sẽ được gửi kèm theo bài báo cáo này. 
% \end{itemize}
\noindent Giống như đồ thị, để tiết kiệm bộ nhớ thì thay vì sử dụng ma trận kề, các lập trình viên thường sử dụng lớp vector (trong C++). Tương tự với mô hình trên, chúng ta sẽ dùng các dictionary để lưu trữ cấc trạng thấi và tính toán các phân phối. Chi tiết code sẽ được gửi kèm theo bài báo cáo này. 
 
%%%%%% Anh Tu && Le Ngu
% \section{Phân tích sự tiến triển của HIV/AIDS}

% \pagebreak
% %%%% Le Ngu
% \section{Chơi cờ tỷ phú}